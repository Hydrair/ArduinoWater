\section{Kommunikation}
Die Kommunikation zwischen dem ESP8266 und dem Arduino Uno erfolgt über UART (Universal Asynchronous Receiver Transmitter).
Über diese serielle Schnittstelle schickt der ESP8266 Befehle und konfigurierbare Parameter.
Der Arduino übermittelt Messwerte, die er von den Feuchtigkeitssensoren liest.
Um das zu ermöglichen wird jeweils der TX Port des einen Controllers mit dem RX Port des anderen verbunden.

\subsection{Funktionen}
Die verwendeten Nachrichten wurden in \ref{arduinokomm} aufgezeigt.
Folgende Funktionen ermöglichen die Kommunikation der beiden Mikrocontroller:
\newline



\begin{tabular}{|rl|p{5.5cm}|}
    \hline
    void & recvWithStartEndMarkers() & Filtert Eingang auf durch $<$ $>$ begrenzte Nachricht.\\\hline
    void & parseData() & Wertet Nachricht aus und extrahiert String und Integer.\\\hline
    void & serialHandler() & Verwaltet die extrahierten Daten.\\\hline
    String & clearFirstEntry(String) & Schiebt den ersten Wert im Array hinaus und füllt es mit dem Neuesten.\\\hline
    void & sendMessage(String, Int) & Gibt Arduino einen Interrupt, der auf Nachricht wartet.\\\hline
\end{tabular}

