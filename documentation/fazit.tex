\newpage
\fancyhf{}
\rhead{Seite \thepage}
\section{Fazit}
Der Aufbau der Wasserversorgung war kniffliger als anfangs erwartet. Vor allem das
Auffinden der richtigen Fittings für die Gewinde mit den richtigen Durchmessern
stellte sich als Problem heraus. Nachdem die Probleme behoben waren,
funktionierte die geplante Versorgung aber einwandfrei und erfüllt die gestellten Anforderungen. 
Beim Aufbau der Energieversorgung ergaben sich nur Probleme beim Löten. So war der DC/DC-Wandler
mit seinen dünnen Stiften schwer einzulöten. Die geplante Schaltung verrichtet ihren Dienst, wie
erwartet. Im Nachhinein fallen noch
vergessene Details auf. So wäre ein Schalter für den Strom außen am Gehäuse von Vorteil gewesen.
Im Moment geht das System an, wenn das Netzteil eingesteckt wird. Auch das Verwenden von direkt
angelöteten Kabeln hat sich als nicht so
vorteilhaft herausgestellt. Steckerleisten wären wesentlich einfacher zu handhaben gewesen. Auch
die Ordnung im Elektrikkasten hätte sich dadurch erhöht.
Das Gehäuse ist für einen Prototypen vollkommen in Ordnung. Anpassbar wären noch passende Löcher
für Schalter und Steckerbuchsen.

Es  wäre  sinnvoller  gewesen,  die  Stromverteiler-  und
Leistungsschaltungsplatine  zu  kombinieren.  Dadurch  hätten  einige
Kabelverbindungen  weggelassen  werden  können  und  die  Übersicht
im  Elektronikkasten  verbessert  werden.  Um  den  Zusammenbau  und
die  Verkabelung  der  einzelnen  Platinen  zu  verbessern  wäre  es
hilfreich  gewesen,  nicht  alle  Kabel  fest  an  die  Platine  zu  löten,
sondern  alle  Verbindungen  mit  Hilfe  von  Stiftleisten  zu  realisieren.
So  hätten  beim  Zusammenbau  nur  die  Kabel  in  die  richtige  Stiftleiste
gesteckt  und  nicht  jedes  einzelne  Kabel  verlötet  werden  müssen.  Am
Ende  wurde  das  verlöten  der  Kabel  recht  umständlich,  da  das  Erreichen
der  Lötstelle  durch  die  vielen  bereits  vorhandenen  Kabel  erschwert  wird.
Zusätzlich  macht  dies  auch  eine  Reparatur  der  Platine  einfacher,  da  nur
alle  Kabel  abgezogen  und  nicht  abgelötet  werden  müssen.  Auch  hätte  die
die  Konstantanspannungsquelle  auf  den  Sensorplatinen  weggelassen  werden
können,  da  wir  die  Spannungsversorgung  mit  5V  extern  realisiert  haben.

Man sollte sich immer einen Überblick über die verfügbaren Pins des
Mikrocontrollers machen, da häufig bestimmte Funktionalitäten und an
bestimmten Pins verfügbar sind. Um im Nachhinein Pins innerhalb des Programms
noch zu ändern, sind Variable Adressen gegenüber festen Adressen sinnvoll.
Bei der Verwendung mehrerer Mikrocontroller ist stets auf ein einheitliches
Bezugspotential zu achten.

Zu Anfang wäre mehr Erfahrung mit dem ESP8266 sehr hilfreich gewesen.
Dadurch dass diese mit einer sehr kryptischen Firmware ausgeliefert werden,
gestalteten sich die ersten Schritte damit als sehr schwer. Doch sobald die alternative
Firmware geflasht wurde, konnte gut auf Fähigkeiten mit C/C++ zurückgegriffen werden.
Auch das Debuggen stellte sich als schwieriger als gedacht heraus. Es war kein Debugger
vorort den man hätte benutzen können, also wurde der Code via Serialausgaben verfolgt.
Dies ist ähnlich bei der Webseite an sich. Da diese in einem String gespeichert wurde und
dies die C-typischen Stolpersteinen war das Hinzufügen und Entfernen von Features herausfordernd.
